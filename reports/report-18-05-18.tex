\documentclass{article}

\begin{document}
  
After a lot of reading on experimental techniques and the kind of data that we have available I came up with two avenues that I propose to go down on. They are both in the scope of DNA and intercalators, but provide \emph{novel} and interesting approaches and techniques.

Experimental values are fundamental and essential for the project to succeed. That is because they provide a concrete aim that we can work towards. Currently the aim of the project was too broadly defined, and understanding these experimental techniques was really helpful in getting a better grasp of what I want to do in the project.

\section{Avenue 1: AMES test, QSAR, ligand trajectory featurisation and mutagenicity prediction}

As Dr. Arabi have noted before, the AMES test provides a binary classification of planar aromatic intercalators of whether they are mutagen or not. Fortunately there is a large (~5000) set of data points that can be used to build a classifier for this.

we would start out with QSAR features that are generated by rdkit as input for the classifier, and we would build up to more advanced feature set like data from simulation of the ligand only! Note we have 5000 ligands, so a small water box and short simulation is what we would aim for.

\section{Avenue 2: Absolute binding affinity + DNA pulling experiment with intercalators}

A relatively new experiment based on force spectroscopy method provide a molecular level insight into DNA intercalator interactions. 
  
\end{document}