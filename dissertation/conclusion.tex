\chapter{Conclusion}

Calculating the binding affinity of intercalators bound to between DNA basepairs can be achieved with increasing levels of accuracy by computational free energy methods. 

By using fast docking and scoring techniques, we were able to differentiate intercalators with changing number of rings in the planar intercalating scaffold. The ESMACS protocol allowed us to rank a congeneric series of intercalators with high correlation to experiment. Finally the TIES method was able to predict relative binding free energies with an RMSE of just 0.41 kcal/mol. 

There is an increasing number of therapeutics solutions for treating cancer that rely of drugs acting on DNA. Being able to calculate the binding affinity of molecules when intercalated between DNA basepairs in silico, allows us to design and develop new compounds that have even higher efficacy than current ones. Combining the protocols developed here with the workflow management tools, we will be able to run simulation campaigns, probing hundreds of molecules and optimising for efficacy to find the most potent intercalators.

Future work will include the further optimisation of these protocols, as well investigating the dependence of the binding affinity on the specific sequence where the intercalation happens and looking at the mechanism of intercalation known to be composed of multiple transition states. We will also collect more experimental binding free energy values and run simulations for a wider range of intercalators. It will be important to compare results to experiment where the free energy difference between intercalators is larger than the error on the estimates. 
