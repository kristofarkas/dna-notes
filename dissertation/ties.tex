\section{TIES}

The complexes of target DNA and hybrid intercalators were solvated in an orthorhombic water box with buffer width of 11 angstrom. The system was neutralized electrostatically with counterions. The BSC1\cite{ivani2016parmbsc1} force field was employed in all our simulations for DNA parameters. Intercalator parameters were produced using the general AMBER force field 2 (GAFF2) \cite{wang2004development}. The AM1BCC \cite{jakalian2002fast} %
semi-empirical methods was used to produce the partial charges on the intercalator atoms (AmberTools 18 \cite{ambertools18}) after geometry optimization with the OpenEye OEGauss toolkit. The simulation engine NAMD 2.12 was used for producing all the trajectories of the complex systems with periodic boundary conditions. The system was brought to and maintained at a temperature of 300 K and 1 atmosphere using the NAMD implementation of the Langevin thermostat (with a damping coefficient of $5 ps^{-1}$) and a Berendsen barostat (compressibility of $4.57 \times 10^{-5} bar^{-1}$ and a relaxation time of $100 fs$). The time step was set to $2 fs$. The van der walls terms perturbed linearly with respect to $\lambda$. To avoid end point catastrophes were atoms near the end point appear suddenly too close to each other a soft core potential was used for the van der Walls interactions. The electrostatic interactions of the disappearing atoms are linearly decoupled until $\lambda = 0.55$ then completely turned off afterwards, and those atoms that are appearing are electrostatically turned on from $\lambda = 0.45$ linearly increasing until the end.

All calculations were done with 13 $\lambda$ windows. At each value of $\lambda$ 5 replica simulations were run to assess the error. For each replica, the standard protocol for minimisation and equilibration was performed, that is 4 ns of simulation time. Production runs for each replica were 16 ns long. While the coordinates were recorded every 10 ps, $\partial V / \partial \lambda$ values were recorded every 2 ps. The choice of 16 ns for the simulation length and 5 for the ensemble size is based on the uncertainty quantification and error analysis discussed in the previous work. The protocol described here can be adjusted to the specific system at hand, but this was enough to converge results for all of the systems in this study. The size of the ensemble can be adjusted even at specific windows to account for additionally uncertainty that may arise. The TIES workflow can be executed in an embarrassingly parallel fashion, and given sufficient resources the simulations can finish in 15 hours. In general the turnaround time depends on the system size, number of cores used per simulation instance with GPUs offering additional speedup.