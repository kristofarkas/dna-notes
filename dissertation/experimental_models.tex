\subsection{Experimental models}

The structure of DNA and other biological molecules can be determined in a number of ways including X-ray, NMR and Cryo-EM \cite{arnott1973refinement}. Unfortunately, it is not always possible to determine the structure of the desired system under investigation experimentally. In that case it might be the case that we can use our knowledge of similar structure and modelling to generate new structures that are likely to resemble reality. Here we briefly describe these techniques.

\subsubsection{X-ray crystallography}

The most notable method, especially from a historical point of view, to analyse the structure of DNA is by X-ray diffraction \cite{branden1999introduction}. This method lead to the discovery of the basic double helical structure of the DNA strand. The resolution is determined by the wavelength of the radiation used to investigate the structure. In case of biologically relevant molecules we want to distinguish between atoms, therefore a sub 1 angstrom resolution is required. This resolution corresponds to a wavelength of X-ray radiation. To produce a high quality image a regular array of units to scatter the radiation is required. This means that we have to crystallise the molecule, which places it into a state somewhat different from the biological conditions it was in originally.

