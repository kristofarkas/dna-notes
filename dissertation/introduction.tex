\section{DNA}

\subsection{DNA intercalation}

Shortly after the unraveling of the complex structure of the double stranded DNA that self-assembles into a helical form started the exploration of ligands that form complexes the DNA in a diverse set of binding forms. Certain molecules could bind to DNA grooves, or intercalate between DNA basepairs. Of these two main categories of binding, the former is less of interest or studied. Intercalators on the other hand have a wide range of interest and use cases, including cancer treatment, due to their varied binding properties to DNA. The existence of intercalators was first predicted and later confirmed by Lerman et al. whereby a small planar aromatic molecule inserts between consecutive DNA-base pairs, perturbing the local structure of the DNA, but still having a stabilising effect through base pair stacking. Intercalation can potentially explain, and aid in understanding a number of DNA specific phenomena including transcription, mutation, and the mechanism of certain anti-tumour agents, like the anti-cancer agent small molecule acridine. Intercalator can be neutrally charged or cationic, they disrupt the continuous and predictable flow of the genetic information encoded in the base-pairs. In contrast other forms of DNA binding, major or minor groove, or electrostatic binding, does not deform the structure, and has a smaller effect of the function of DNA. Intercalation greatly changes the structure of DNA, including the base pair distance, decreasing the twist angle by up to 50\%, essentially unwinding the DNA to some extent in the local neighbourhood of the intercalation site. In order to accommodate the molecule, the inter-basepair distance has to be increased, sometimes up to double its original size. The increase in separation elongates the whole DNA helix. Moreover while the DNA helix unwinding compensates for the insertion maximising the base stacking and hence the stability, the backbone structure, for example the distance between consecutive phosphate groups remains nearly identical. This level of perturbation and the fact that it is reversible (as the binding is non-covalent) provides a basis for a wide range of applications. 

% Application and importance in drug treatment of intercalators

The importance of intercalators in drug design and the potential application of these to treat a wide range of cancer has already been shown by certain compounds and FDA approved treatments. Two similar compounds, doxorubicin and daunomycin, are currently used as anti cancer treatment for more forms of cancer than any other compound. Intercalation of these compounds disrupts replication and transcription of DNA and leads to the death of cancerous cells. Once the crystal structure of this complex was determined with X-ray diffraction methods, the mode of action was reavealed to be that of intercalation, as the fused ring system is indeed sandwiched between basepairs. The mechanism was theorised to be a ring-insertion mechanism with the inter-basepair distance increasing, but certain structural features staying the same for example the hydrogen bonding.

% Talk something about the kinetics including the metadyanmics based approaches. This is starting to place out study in the context of the larger fiels. Thank you! 

The basepair distance opening and molecule insertion nonetheless is too simple of mechanism, and certain high quality experimental measurements indicate a more sophisticated binding mechanism. For example, kinetic measurements indicate a three step intercalation process, with the first step being a fast binding to the outside of the DNA chain, and the second and third slower conformational adjustments. However the experiments are low resolution and the hypothesis of the intercalations is not fully elucidated. Certain computational techniques have been used to investigate this mechanism. Metadynamics and umbrella sampling can provide atomic-scale resolutions and free energy estimates of the intercalation mechanism. They confirm the earlier studies, that the process is composed of three steps, an initial binding in the minor groove of the DNA, followed by the rotation of the drug and DNA deformation into a seconds intermediate metastable state, and finally separated by only a small barrier complete rotation of the molecule into the space between basepairs to establish a full basepair stacking interaction.

% 



